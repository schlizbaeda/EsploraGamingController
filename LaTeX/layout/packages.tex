% Adjustment of page layout ----------------------------------------------------
%   see style.tex
% ------------------------------------------------------------------------------
\usepackage[
    automark, % Kapitelangaben in Kopfzeile automatisch erstellen
    headsepline, % Trennlinie unter Kopfzeile
    ilines % Trennlinie linksb�ndig ausrichten
]{scrpage2}

% Umlauts ----------------------------------------------------------------------
%   Set CodePage:
%   Use special characters like German umlauts (����)  directly in the LaTeX
%   source. Enables correct hyphenation rules for words containing umlauts.
% ------------------------------------------------------------------------------
\usepackage[latin1]{inputenc}
%\usepackage[utf8]{inputenc}
\usepackage[T1]{fontenc}
\usepackage{textcomp} % Euro char, Copyright etc.

% font -------------------------------------------------------------------------
\usepackage{lmodern} % better fonts
\usepackage{relsize} % Set relative font size

% Graphics and Pictures --------------------------------------------------------
% Enable including of JPG files
\usepackage[dvips,final]{graphicx}
% relative graphics path
\graphicspath{{pics/}}

% Commands from AMSTeX for mathematical symbols as \boldsymbol \mathbb ---------
\usepackage{amsmath,amsfonts}

% Index output with \printindex ------------------------------------------------
\usepackage{makeidx}

% Easy definition of line spacing, page margins etc. ---------------------------
\usepackage{setspace}
\usepackage{geometry}

% ------------------------------------------------------------------------------
% --- index of abbreviations ---------------------------------------------------
% ------------------------------------------------------------------------------
% Create symbol directories easily. Based on MakeIndex:
%  makeindex.exe %Name%.nlo -s nomencl.is -o %Name%.nls
% creates the directory. This command can be used e.g. in TeXnicCenter
% as a postprocessor, so that it needn't to be used manually all the time.
% The definitions are moved into the file "Glossar.tex".
% ------------------------------------------------------------------------------
\usepackage[intoc]{nomencl}
\let\abbrev\nomenclature
\renewcommand{\nomname}{Abk�rzungsverzeichnis}
\setlength{\nomlabelwidth}{.25\hsize}
\renewcommand{\nomlabel}[1]{#1 \dotfill}
\setlength{\nomitemsep}{-\parsep}

\usepackage{acronym}


% Text flow around images ------------------------------------------------------
\usepackage{floatflt}

% ------------------------------------------------------------------------------
% --- Listings to include source code ------------------------------------------
% ------------------------------------------------------------------------------
\usepackage{listings}
\usepackage[table]{xcolor} 
% colours for syntax colouring / alternatively: {RGB}{0-255,0-255.0-255}
%%%%\definecolor{hellgelb}{rgb}{1,1,0.9}			     
\definecolor{colKeys}{rgb}{0,0,1}
\definecolor{colIdentifier}{rgb}{0,0,0}
\definecolor{colComments}{rgb}{0,0.5,0.1} 
\definecolor{colString}{rgb}{1,0,0}
\lstset{
    float=hbp,
    basicstyle=\ttfamily\color{black}\small\smaller, % the size of the fonts that are used for the code 
    identifierstyle=\color{colIdentifier},
    keywordstyle=\color{colKeys},
    stringstyle=\color{colString},
    commentstyle=\color{colComments},
    columns=flexible,
    tabsize=3,
    %frame=single,                     % add frame arround the code
    extendedchars=true,
    showspaces=false,
    showstringspaces=false,
    numbers=left,                      % where to put the line numbers
    numberstyle=\tiny,                 % fontsize for line
    stepnumber= 1,                     % stepnumber between line-numbers
    breaklines=true,                   % sets automatic line breaking
    captionpos=b,                      % sets caption position to bottom
    backgroundcolor=\color{hellgelb},
    breakautoindent=true,
    }

% URL handling -----------------------------------------------------------------
\usepackage{url}

% important for correct citation -----------------------------------------------
\usepackage[numbers]{natbib}  % [square] to [numbers]

% ------------------------------------------------------------------------------
% --- PDF options --------------------------------------------------------------
% ------------------------------------------------------------------------------
\definecolor{darkblue}{rgb}{0,0,0.5} % colour of PDF links
\usepackage[
    bookmarks,
    bookmarksopen=true,
    colorlinks=true,
% colour definition of PDF links
    linkcolor=darkblue,% simple internal links
    anchorcolor=black, % anchor text
    citecolor=blue,    % links to bibliography
    filecolor=magenta, % links to open local files
    menucolor=red,     % Acrobat menu items
    urlcolor=cyan, 
% colour definitions for printing (all in black)
    %linkcolor=black,  % simple internal links
    %anchorcolor=black,% anchor text
    %citecolor=black,  % links to bibliography
    %filecolor=black,  % links to open local files
    %menucolor=black,  % Acrobat menu items
    %urlcolor=black, 
    backref,
    % for correct creation of bookmarks:
    plainpages=false,
    pdfpagelabels,
    hypertexnames=false,
    linktocpage % link to page numbers insteag of text
]{hyperref}
% Commands that output umlauts lead to errors when they are passed as options to hyperref
\hypersetup{
    pdftitle={\titel \untertitel},
    pdfauthor={\autor},
    pdfcreator={\autor},
    pdfsubject={\titel \untertitel},
    pdfkeywords={\titel \untertitel},
}

% Continuous Numbering of foot notes -------------------------------------------
\usepackage{chngcntr}

% for long tables --------------------------------------------------------------
\usepackage{longtable}
\usepackage{array}
\usepackage{ragged2e}
\usepackage{lscape}

% column definitions at right margin with defined width ------------------------
\newcolumntype{w}[1]{>{\raggedleft\hspace{0pt}}p{#1}}

% Change format of lists -------------------------------------------------------
\usepackage{paralist}

% necessary for definition of own commands
\usepackage{ifthen}

% defines commands for \todo and \listoftodos etc.
\usepackage{todonotes}

% ensures that spaces behind parameterless macros are not interpreted as macro end characters
\usepackage{xspace}

% format of figure annotations (captions)
\usepackage{caption}
\captionsetup{labelfont=bf,textfont=it} % figure bold, text italic

% including of message boxes
\usepackage[tikz]{bclogo}

% include underlining of text
\usepackage{ulem}

% format of tables
\usepackage{booktabs}
\renewcommand{\arraystretch}{1} % line spacing inside tables of 2

% including PDF files
\usepackage{pdfpages}
